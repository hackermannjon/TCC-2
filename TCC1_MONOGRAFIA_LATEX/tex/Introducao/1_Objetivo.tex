O presente trabalho de conclusão de curso (TCC) tem como objetivo central explorar e implementar metodologias de inteligência artificial (IA) em um contexto de aplicativo de namoro, com a finalidade de desenvolver um algoritmo de matchmaking mais completo. Este algoritmo não se limitará apenas à recomendação de usuários, mas também à previsão de potenciais usuários para um relacionamento de sucesso e estável. A implementação deste algoritmo será realizada em um contexto real, em um aplicativo de namoro funcional.

\section*{Justificativa do Objetivo}

\begin{enumerate}[label=\arabic*.]
    \item \textbf{Transformação dos Algoritmos de Matchmaking}\\
    Tradicionalmente, os algoritmos de matchmaking em aplicativos de namoro se baseiam em dados demográficos básicos e preferências superficiais. No entanto, a introdução de IA permite uma análise mais profunda e personalizada das preferências dos usuários. Algoritmos de IA, como o "Most Compatible" do Hinge, utilizam aprendizado de máquina para analisar as preferências e ações dos usuários, sugerindo matches altamente compatíveis. Esta abordagem tem demonstrado aumentar significativamente as chances de matches bem-sucedidos, promovendo conexões mais significativas e duradouras.

    \item \textbf{Previsão de Relacionamentos Estáveis}\\
    A aplicação de IA para prever a estabilidade de relacionamentos vai além da simples correspondência de perfis. Estudos mostram que algoritmos podem ser desenvolvidos para analisar aspectos mais profundos, como comportamentos e interações dos usuários, facilitando previsões sobre a durabilidade e sucesso de um relacionamento. Este avanço é crucial para a evolução dos aplicativos de namoro, proporcionando aos usuários não apenas um parceiro potencial, mas um parceiro com alta probabilidade de um relacionamento duradouro e satisfatório.

    \item \textbf{Implementação em um Contexto Real}\\
    Para validar a eficácia do algoritmo desenvolvido, é essencial sua implementação em um ambiente real. Aplicativos como Tinder e Bumble já utilizam IA para melhorar a experiência do usuário, otimizando fotos de perfil e moderando conteúdo para aumentar a segurança. A implementação em um aplicativo funcional permitirá testar e ajustar o algoritmo em tempo real, garantindo sua eficácia e eficiência no mundo real. Além disso, aplicativos como o Match.com utilizam chatbots com IA para sugerir locais de encontro, mostrando o potencial da IA em oferecer uma experiência de usuário mais completa e personalizada.
\end{enumerate}