\subsection{Influência de Conhecidos em Comum}
\subsubsection{2.2.1 Teoria dos Laços Fracos de Granovetter}
A Teoria dos Laços Fracos, proposta por Mark Granovetter em 1973, é uma das contribuições mais influentes para o estudo das redes sociais. Esta teoria sugere que os laços fracos, que são conexões mais distantes ou menos frequentes, desempenham um papel crucial na difusão de informações e na formação de novas oportunidades dentro de uma rede social. Granovetter argumentou que, enquanto laços fortes (como familiares e amigos próximos) são importantes para o apoio emocional, são os laços fracos que introduzem informações novas e diversas, conectando indivíduos a diferentes redes sociais e possibilitando o acesso a novos grupos sociais \textcolor{blue}{[\cite{Granovetter1973}]}.

Em um contexto de relacionamentos, laços fracos, como conhecidos em comum, podem atuar como pontes entre diferentes círculos sociais, aumentando a probabilidade de encontros entre indivíduos que, de outra forma, não teriam se cruzado. A presença de conhecidos em comum pode facilitar a confiança inicial e fornecer um ponto de partida para interações significativas. Estudos empíricos, como os realizados na plataforma LinkedIn, demonstraram que laços fracos são especialmente eficazes para facilitar a mobilidade social e profissional, ilustrando seu potencial para também promover novos relacionamentos pessoais \textcolor{blue}{[\cite{Granovetter1973}]}.

Granovetter também destaca que, em ambientes onde a troca de informações é vital, como em economias digitais e setores de alta tecnologia, a importância dos laços fracos é amplificada. Nesse sentido, os aplicativos de namoro podem se beneficiar ao identificar e utilizar laços fracos para recomendar parceiros potenciais, criando oportunidades para conexões que de outra forma não ocorreriam. A teoria dos laços fracos sugere que a introdução de novos contatos através de conhecidos em comum não apenas amplia as redes sociais, mas também enriquece as experiências sociais, oferecendo aos indivíduos uma chance de explorar novas dinâmicas interpessoais \textcolor{blue}{[\cite{Granovetter1973}]}.

\subsubsection{2.2.2 Teoria do Suporte Social}
A Teoria do Suporte Social é uma estrutura conceitual amplamente estudada que examina como as redes sociais proporcionam apoio emocional, instrumental e informacional aos indivíduos. Essa teoria sugere que a presença de suporte social robusto pode atuar como um amortecedor contra o estresse e promover o bem-estar psicológico e físico. Conhecidos em comum são um componente vital dessas redes de suporte, fornecendo um contexto social que facilita a troca de apoio e validação entre parceiros \textcolor{blue}{[\cite{Cohen2004}]}.

Empiricamente, a teoria do suporte social é apoiada por numerosos estudos que demonstram que indivíduos com redes sociais fortes, incluindo conhecidos em comum, relatam maior satisfação e estabilidade em seus relacionamentos. Essas redes oferecem um ambiente de confiança e segurança, essencial para o desenvolvimento de relações íntimas e duradouras. A presença de amigos em comum pode atuar como um mediador em conflitos, oferecendo conselhos imparciais e suporte prático que reforçam os laços entre parceiros \textcolor{blue}{[\cite{Cohen2004}]}.

Além disso, a teoria destaca a importância da reciprocidade nas interações sociais, onde a troca mútua de apoio fortalece as conexões e promove a resiliência do relacionamento. Em um cenário de aplicativos de namoro, a integração de conhecidos em comum pode não apenas aumentar a confiança inicial entre os usuários, mas também criar uma rede de suporte que encoraje interações mais significativas e sustentáveis. O suporte social proporcionado por conhecidos em comum pode, portanto, desempenhar um papel crucial na facilitação e manutenção de relacionamentos estáveis \textcolor{blue}{[\cite{Cohen2004}]}.

\subsubsection{2.2.3 Teoria do Contágio Social}
A Teoria do Contágio Social explora como comportamentos, emoções e normas são transmitidos através de redes sociais, influenciando os indivíduos de maneira direta e indireta. Conhecidos em comum podem servir como catalisadores para o contágio social, promovendo a adoção de comportamentos positivos e normas sociais que favorecem a formação e o fortalecimento de relacionamentos \textcolor{blue}{[\cite{Christakis2007}]}.

De acordo com a teoria, quando duas pessoas compartilham amigos, há uma maior probabilidade de que atitudes e comportamentos positivos sejam reforçados, criando um ambiente mais propício para o desenvolvimento de laços estáveis. Esse fenômeno pode ser observado em várias áreas, desde a saúde mental até o comportamento organizacional, onde o contágio social ajuda a difundir práticas saudáveis e a promover a coesão social \textcolor{blue}{[\cite{Christakis2007}]}.

Em relacionamentos, o contágio social pode facilitar a convergência de valores e expectativas entre os parceiros, aumentando a compatibilidade e a satisfação. Conhecidos em comum podem atuar como modelos de comportamento, incentivando práticas que fortalecem a relação, como a comunicação aberta e o apoio mútuo. Nos aplicativos de namoro, considerar o contágio social através de conhecidos em comum pode ajudar a identificar matches com maior potencial de sucesso, ao alinhar usuários com normas e comportamentos compatíveis \textcolor{blue}{[\cite{Christakis2007}]}.

\subsubsection{2.2.4 Aplicativos de Sucesso Utilizando Conhecidos em Comum}
O uso de conhecidos em comum como ferramenta de matchmaking tem se mostrado eficaz em vários aplicativos de namoro de sucesso. Exemplos notáveis incluem Tinder, Hinge e Bumble, cada um integrando essa funcionalidade de maneira a enriquecer a experiência do usuário e promover conexões mais significativas \textcolor{blue}{[\cite{DatingApps2024}]}.

Tinder utiliza amigos em comum para aumentar a confiança e facilitar a introdução entre potenciais matches. Através da integração com redes sociais como o Facebook, Tinder oferece aos usuários a capacidade de ver amigos em comum, fornecendo uma base de familiaridade que pode encorajar interações iniciais. Este recurso é reforçado pelo "Tinder Matchmaker", onde amigos podem recomendar perfis uns aos outros, destacando a importância das redes sociais compartilhadas na formação de conexões \textcolor{blue}{[\cite{DatingApps2024}]}.

Hinge também adota conhecidos em comum como uma parte central de seu algoritmo de matchmaking. Ao focar em conexões significativas, Hinge permite que os usuários identifiquem amigos compartilhados com possíveis matches, criando uma rede de suporte que pode validar e encorajar novas interações. Este enfoque em profundidade e compatibilidade é parte do que faz de Hinge uma escolha popular para aqueles que procuram relacionamentos sérios e duradouros \textcolor{blue}{[\cite{DatingApps2024}]}.

Bumble incorpora conhecidos em comum para promover um ambiente seguro e autêntico. Ao usar conexões sociais compartilhadas para sugerir possíveis matches, Bumble cria uma rede onde a confiança e a familiaridade são fundamentais. Isso é especialmente importante para usuários que buscam interações respeitosas e genuínas, alinhando-se com a missão do aplicativo de empoderar as mulheres e criar conexões mais significativas \textcolor{blue}{[\cite{DatingApps2024}]}. 
