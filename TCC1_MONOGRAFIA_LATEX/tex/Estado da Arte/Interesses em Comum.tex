\subsubsection{2.1.1 Efeito de Similaridade-Atração}
O efeito de similaridade-atração é um fenômeno amplamente estudado na psicologia social que sugere que as pessoas são mais propensas a se sentir atraídas por indivíduos que compartilham características semelhantes, como atitudes, interesses e valores. Essa teoria se fundamenta na ideia de que a similaridade proporciona uma base de entendimento mútuo e validação social, elementos cruciais para o desenvolvimento e manutenção de relações interpessoais bem-sucedidas \textcolor{blue}{[\cite{Berscheid1998}]}.

Os estudos sobre o efeito de similaridade-atração indicam que a semelhança entre indivíduos facilita interações mais agradáveis e reforça a percepção de aceitação e pertencimento. Isso ocorre porque os indivíduos são mais propensos a esperar rejeição por parte de pessoas que são dissimilares, enquanto a interação com semelhantes tende a ser mais confortável e agradável \textcolor{blue}{[\cite{Montoya2008}]}. Além disso, a similaridade aumenta a probabilidade de encontros casuais em ambientes sociais comuns, como eventos esportivos ou atividades culturais, onde interesses compartilhados são evidentes \textcolor{blue}{[\cite{Newcomb1961}]}.

Outra explicação para o efeito de similaridade-atração é o reforço social, onde interações com indivíduos semelhantes são mais gratificantes, pois reforçam as próprias crenças e atitudes de cada indivíduo. Esse reforço ocorre tanto em contextos de amizade quanto em relacionamentos românticos, onde a validação mútua fortalece o vínculo \textcolor{blue}{[\cite{Byrne1971}]}. Além disso, pesquisas demonstram que, em plataformas de namoro online, perfis que exibem maior similaridade percebida tendem a receber mais interações positivas \textcolor{blue}{[\cite{Montoya2008}]}.

Por fim, é importante considerar que, embora a similaridade possa inicialmente atrair parceiros, a percepção contínua de semelhança pode ser necessária para manter a atração ao longo do tempo. Discrepâncias percebidas em áreas centrais podem levar a desentendimentos, enquanto a manutenção de um senso de similaridade, mesmo em áreas onde as opiniões podem divergir, ajuda a sustentar relacionamentos saudáveis \textcolor{blue}{[\cite{Montoya2008}]}.

\subsubsection{2.1.2 Raciocínio Autoessencialista}
O raciocínio autoessencialista é uma abordagem teórica que propõe que a atração por indivíduos semelhantes ocorre devido a uma percepção de essência compartilhada. Esse conceito sugere que as pessoas tendem a categorizar aqueles que compartilham atributos semelhantes como "pessoas como eu", aplicando uma essência comum que facilita a identificação de uma realidade compartilhada \textcolor{blue}{[\cite{Chu2023}]}.

Pesquisas indicam que o raciocínio autoessencialista desempenha um papel fundamental na forma como as pessoas percebem e interpretam a similaridade em relacionamentos interpessoais. Estudos experimentais mostram que indivíduos com crenças autoessencialistas mais fortes experimentam uma maior atração por pessoas semelhantes, pois veem essas semelhanças como indicativas de um entendimento comum do mundo \textcolor{blue}{[\cite{Montoya2013}]}.

A aplicação desse raciocínio em relacionamentos românticos é especialmente relevante, pois ajuda a explicar por que interesses e valores compartilhados podem levar a conexões mais profundas e satisfatórias. Quando as pessoas percebem que compartilham uma essência com seu parceiro, isso reforça a conexão emocional e a compreensão mútua, elementos essenciais para a construção de um relacionamento duradouro \textcolor{blue}{[\cite{Montoya2008}]}.

Além disso, o raciocínio autoessencialista sugere que a similaridade não é apenas uma questão de compartilhamento de características superficiais, mas também de um alinhamento mais profundo nas formas de ver e entender o mundo. Isso pode ser particularmente importante na resolução de conflitos, onde a percepção de uma realidade compartilhada ajuda os parceiros a navegarem por desafios e diferenças de maneira mais harmoniosa \textcolor{blue}{[\cite{Chu2023}]}.

Essa abordagem também destaca a importância de cultivar uma percepção de similaridade em áreas centrais de um relacionamento, ao mesmo tempo em que reconhece e valoriza as diferenças. Ao aplicar o raciocínio autoessencialista, os aplicativos de namoro podem aprimorar suas recomendações, focando em interesses e valores fundamentais que promovem uma sensação de conexão essencial entre os usuários \textcolor{blue}{[\cite{Montoya2013}]}.

\subsubsection{2.1.3 Importância do Compartilhamento de Experiências}
A importância do compartilhamento de experiências em relacionamentos interpessoais é uma área de crescente interesse na psicologia social, destacando como essas experiências contribuem para o fortalecimento das conexões emocionais e sociais \textcolor{blue}{[\cite{Berscheid1998}]}.

Estudos demonstram que experiências compartilhadas aumentam a percepção de emoções e sensações comuns, o que por sua vez fortalece a conexão emocional entre as pessoas envolvidas \textcolor{blue}{[\cite{Byrne1971}]}. Isso é evidenciado pela ativação dos circuitos de recompensa no cérebro durante experiências compartilhadas, como assistir a um filme ou participar de atividades conjuntas, indicando que o simples ato de compartilhar pode intensificar a experiência emocional \textcolor{blue}{[\cite{Montoya2008}]}.

Além disso, as experiências compartilhadas fornecem uma base comum que facilita a comunicação e a compreensão mútua. Elas servem como pontos de referência que podem ser revisitados e discutidos, criando uma narrativa compartilhada que fortalece a identidade do relacionamento \textcolor{blue}{[\cite{Montoya2013}]}. O impacto emocional dessas experiências é amplificado quando os indivíduos sentem que suas percepções e reações estão alinhadas com as dos outros, promovendo um senso de pertencimento e validação \textcolor{blue}{[\cite{Montoya2008}]}.

Em um contexto de aplicativos de namoro, incentivar encontros e atividades que promovam o compartilhamento de experiências pode melhorar a eficácia dos algoritmos de matchmaking. Ao conectar usuários com base em interesses e atividades comuns, os aplicativos podem facilitar o desenvolvimento de relações mais profundas e significativas \textcolor{blue}{[\cite{Chu2023}]}.

A promoção de experiências compartilhadas não se limita apenas a eventos agradáveis; enfrentar desafios e superar obstáculos juntos também pode fortalecer o vínculo entre os parceiros. Isso ocorre porque a superação de dificuldades em conjunto reforça a confiança e demonstra a capacidade de colaboração, aspectos essenciais para a resiliência de um relacionamento \textcolor{blue}{[\cite{Montoya2008}]}.
