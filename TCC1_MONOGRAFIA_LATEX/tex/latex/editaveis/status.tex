\chapter[Status do Trabalho]{Status do Trabalho}
\label{cap:status}

Este capítulo aborda o \textit{status} atual do trabalho relacionado à primeira etapa do 
TCC. Inicialmente é apresentado o Andamento das Atividades (\ref{section:andamento}) previstas 
no cronograma descrito na Metodologia (\ref{section:cronograma}), seguido dos Resultados Obtidos 
(\ref{section:resultados}) até o presente momento. Por fim, são apresentadas as 
Atividades da Segunda Etapa (\ref{section:atividades_seg_etapa}), previstas para o TCC 2 e as 
Considerações Finais do Capítulo (\ref{section:consideracoes_finais_status}).

\section{Andamento das Atividades}
\label{section:andamento}

A execução das tarefas planejadas na fase inicial do Trabalho de Conclusão de Curso 
foi finalizada de acordo com o cronograma delineado na seção \ref{section:cronograma}. 
A Tabela \ref{tab:andamento_tcc1} fornece informações sobre as atividades e subprocessos, 
incluindo seus status correspondentes.

\begin{table}[h]
    \centering
    \caption{Andamento de atividades/subprocessos da primeira etapa do TCC}
    \begin{tabularx}{\linewidth}{l*{5}{>{\centering\arraybackslash}X}}
        \toprule
        \textbf{Atividades/Subprocessos} & \textbf{Andamento} \\
        \midrule
        \rowcolor{gray!20} Definir o Tema & Concluído \\
        Conduzir a Pesquisa Bibliográfica & Concluído \\
        \rowcolor{gray!20} Formular a Proposta Inicial & Concluído \\
        Desenvolver o Referencial Teórico & Concluído \\
        \rowcolor{gray!20} Estabelecer o Suporte Tecnológico & Concluído \\
        Desenvolver a Metodologia & Concluído \\
        \rowcolor{gray!20} Refinar a Proposta & Concluído \\
        Criar a Prova de Conceito Inicial & Concluído \\
        \rowcolor{gray!20} Descrever os Resultados Parciais & Concluído \\
        Apresentar o TCC1 & Pendente \\
        \bottomrule
    \end{tabularx}
    \parbox{\linewidth}{\centering FONTE: Autores}
    \label{tab:andamento_tcc1}
\end{table}

\section{Resultados Obtidos}
\label{section:resultados}

A primeira etapa do TCC alcançou os seguintes objetivos específicos, previstos no Capítulo \ref{cap:introducao} - Introdução:

\begin{itemize}
  \item Estudo sobre Engenharia Reversa de Aplicativos Móveis que se encontram na etapa de Manutenção Evolutiva;
  \item Estudo sobre Reengenharia de Aplicativos Móveis que se encontram na etapa de Manutenção Evolutiva;
  \item Levantamento sobre TDD;
  \item Levantamento sobre DDD;
  \item Levantamento sobre Técnicas de Programação, e
  \item Documentação das principais recomendações da Engenharia de Software acordadas nos estudos e levantamentos realizados.
\end{itemize}

Os objetivos acima foram alcançados por meio da Metodologia Investigativa \ref{section:metod_investigativa} utilizada em 
alguns artefatos produzidos no trabalho. Dentre estes artefatos, vale ressaltar:

\begin{itemize}
  \item Capítulo \ref{cap:referencial}, que aborda o Referencial Teórico do trabalho, realizando um estudo sobre os temas de \textit{Design} de \textit{Software}, Técnicas de Programação, Desenvolvimento \textit{Mobile}, Engenharia Reversa e Reengenhari, temas estes relacionados com a área de contribuição deste TCC.
	\item Capítulo \ref{cap:suporte}, que aborda o Suporte Tecnológico do trabalho, apresentando as principais ferramentas e tecnologias necessárias para a elaboração teórica e prática deste TCC.
	\item Capítulo \ref{cap:metodologia}, que aborda a Metodologia do trabalho, definindo as metodologias específicas que compõem o desenvolvimento teórico e prático deste TCC. Este artefato abordou a Metodologia Investigativa para a pesquisa bibliográfica do trabalho, além das metodologias relacionadas ao desenvolvimento, provas de conceito e análise de resultados.
\end{itemize}

Além destes artefatos, também vale ressaltar o Capítulo \ref{cap:proposta} que definiu as 
provas de conceito que serão desenvolvidas neste trabalho, que compõem a parte 
prática deste trabalho. Nesta etapa do TCC, desenvolveu-se a POC 1 \ref{section:poc_1}, alcançando 
parcialmente os seguintes objetivos específicos:

\begin{itemize}
  \item Aplicação das principais recomendações no Aplicativo Mia Ajuda, e
  \item Exposição dos resultados obtidos.
\end{itemize}

Espera-se concluir totalmente os objetivos acima com o desenvolvimento das demais POCs 
definidas no Capítulo \ref{cap:proposta}, que serão abordadas na segunda etapa deste trabalho.


\section{Atividades da Segunda Etapa}
\label{section:atividades_seg_etapa}

Após a conclusão das atividades previstas para a primeira etapa do trabalho e a aprovação no TCC 1, prevê-se 
a realização das atividades propostas para a segunda etapa, conforme a Tabela \ref{tab:andamento_tcc2}. Em conjunto, 
espera-se elaborar essas atividades utilizando o Suporte Tecnológico, levantado em \ref{cap:suporte}. Por fim, 
é esperado o desenvolvimento de todas as provas de conceito descritas na Proposta (\ref{cap:proposta}).

\begin{table}[h]
    \centering
    \caption{Andamento de atividades/subprocessos da segunda etapa do TCC}
    \begin{tabularx}{\linewidth}{l*{5}{>{\centering\arraybackslash}X}}
        \toprule
        \textbf{Atividades/Subprocessos} & \textbf{Andamento} \\
        \midrule
        Implementar Ajustes & Pendente \\
        \rowcolor{gray!20} Realização das Atividades de Desenvolvimento & Pendente \\
        Realização da Análise de Resultados & Pendente \\
        \rowcolor{gray!20} Finalizar a Monografia & Pendente \\
        Apresentar o TCC2 & Pendente \\
        \bottomrule
    \end{tabularx}
    \parbox{\linewidth}{\centering FONTE: Autores}
    \label{tab:andamento_tcc2}
\end{table}

\section{Considerações Finais do Capítulo}
\label{section:consideracoes_finais_status}

Este capítulo teve o intuito de apresentar o \textit{status} do trabalho referente à primeira etapa do TCC, 
verificando o andamento das atividades previstas no cronograma, bem como seus resultados até o presente 
momento, com base nos objetivos específicos desse trabalho. Por fim, foram apresentadas as atividades 
previstas para o desenvolvimento da segunda etapa do TCC.

Conclui-se, mesmo que de maneira parcial, a respeito da relevância do trabalho, destacando os resultados 
alcançados até o momento, especialmente em relação a trabalhos com temas voltados para a área de Reengenharia 
e a combinação das diversas técnicas de programação (TDD e DDD no caso desse trabalho).
