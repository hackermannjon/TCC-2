Neste estudo, pretendemos explorar como a IA pode ser aplicada para analisar a viabilidade de relacionamentos a partir de aplicativos de namoro, ou seja, a capacidade de identificar padrões que possam prever e recomendar parceiros em plataformas de namoro cujo match tem uma probabilidade maior de se estender em um relacionamento estável e duradouro. O desenvolvimento da tecnologia aumentou a eficácia dessas plataformas pelo aprimoramento dos algoritmos de personalização, e
se tornou essencial para manter a competitividade atual do mercado, além de ser vital para muitos na busca por um parceiro. Embora tenham obtido resultados comerciais impressionantes, desafios como a identificação correta de compatibilidade e vieses nos algoritmos ainda podem ser aprimorados. Considerando a complexidade do problema, é necessário desenvolver soluções que atendam às tendências gerais e pessoais simultaneamente. Este estudo destaca a relevância da proposta para toda a humanidade, ao considerar o uso eficiente da IA em uma análise de padrões de comportamento na interação das plataformas de namoro. Atualmente, o processo de recomendação foca na popularidade do usuário no aplicativo e nos interesses compartilhados entre as pessoas. O estudo tenta adicionar e enfatizar variáveis psicológicas e comportamentais para recomendar matches que geram relacionamentos duráveis.
Portanto, a proposta de IA poderá melhorar o processo de recomendação de matches para usuários de aplicativos de namoro.