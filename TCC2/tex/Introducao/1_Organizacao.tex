\begin{enumerate}
    \item \textbf{Estado da Arte}
    \begin{enumerate}
        \item \textbf{Introdução}
        
        A introdução deste capítulo servirá para definir e explicar os padrões comportamentais mais relevantes para um relacionamento de sucesso. Serão analisados estudos científicos e, além disso, serão investigados padrões comportamentais que são considerados vantajosos e utilizados em aplicativos de relacionamento existentes.

        \item \textbf{Interesses em Comum}
        
        Esta seção irá explorar a influência da personalidade nos relacionamentos. Será discutido como características pessoais, traços psicológicos e interesses em comum podem ser usados para prever a compatibilidade entre usuários. A análise incluirá a revisão de modelos teóricos e estudos empíricos que demonstram a importância da personalidade para a formação de relações duradouras.

        \item \textbf{Conhecidos em Comum (Grafo de Rede Social)}
        
        Nesta parte, a análise se concentrará em como as conexões sociais afetam a probabilidade de sucesso em aplicativos de namoro. Utilizando o conceito de grafos de redes sociais, será discutido o impacto de conhecidos em comum na facilitação de encontros mais confiáveis e com maior potencial de sucesso. Estudos de caso e pesquisas acadêmicas que analisam essa dinâmica serão apresentados.

        \item \textbf{Preferência Visual}
        
        O objetivo desta seção é examinar o papel das preferências visuais no processo de formação de relacionamentos. Será avaliado como os algoritmos podem ser treinados para reconhecer padrões visuais que indicam preferências estéticas e como essas preferências influenciam a atração e a decisão de iniciar um contato.
    \end{enumerate}

    \item \textbf{Fundamentação Teórica}
    \begin{enumerate}
        \item \textbf{Convolutional Neural Networks (CNNs)}
        
        Explora o uso de CNNs para extrair características visuais de perfis, melhorando a recomendação de parceiros ao identificar preferências visuais através de imagens.

        \item \textbf{Transformers}
        
        Discute como Transformers utilizam autoatenção para analisar descrições de perfis e mensagens, capturando interesses e compatibilidades em dados sequenciais.

        \item \textbf{Graph Neural Networks (GNNs)}
        
        Aborda como GNNs modelam conexões sociais, propagando informações entre nós para identificar padrões de interação e fortalecer recomendações.

        \item \textbf{Graph Neural Networks for Social Recommendation (GraphRec) and Adaptation}
        
        Analisa o GraphRec, combinando GNNs e atenção para integrar dados de grafos sociais e de itens, com adaptações para incluir traços de personalidade e características visuais.
    \end{enumerate}

    \item \textbf{Situação Atual}
    \begin{enumerate}
        \item \textbf{Progresso Atual}
        
        Aqui será apresentado um resumo das atividades realizadas, com ênfase nos resultados obtidos até o momento. Serão destacadas as etapas concluídas e as descobertas mais significativas do projeto.

        \item \textbf{Desafios e Dificuldades Encontradas}
        
        A discussão nesta seção abordará os principais desafios técnicos e teóricos enfrentados durante o desenvolvimento do TCC, bem como as estratégias adotadas para superá-los.

        \item \textbf{Próximos Passos}
        
        Será identificada e discutida a sequência de etapas futuras necessárias para a conclusão do projeto. Esta seção destacará as prioridades e planos para finalizar o desenvolvimento do algoritmo e sua implementação.

        \item \textbf{Considerações Finais}
        
        Finalmente, as considerações finais refletirão sobre o trabalho realizado, discutindo as implicações dos resultados para o campo de estudo e propondo direções para pesquisas futuras. A seção também abordará as contribuições do projeto para o avanço dos aplicativos de namoro e suas aplicações práticas.
    \end{enumerate}
\end{enumerate}