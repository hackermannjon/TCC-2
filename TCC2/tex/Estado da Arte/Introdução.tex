Este capítulo tratará dos conceitos teóricos essenciais para a elaboração deste Trabalho de Conclusão de Curso, que busca explorar algoritmos de matchmaking em aplicativos de namoro. Inicialmente, serão discutidos os interesses em comum (2.1), que incluem gostos, valores e atividades compartilhadas entre indivíduos, demonstrando como essas semelhanças influenciam a formação e a durabilidade dos relacionamentos.

Em seguida, o capítulo abordará o conceito de conhecidos em comum (2.2), uma estratégia que explora a rede social dos usuários para identificar conexões mútuas que possam facilitar interações e criar um ambiente de confiança entre os possíveis parceiros. Estudos demonstram que conhecidos em comum podem atuar como uma ponte de confiança, aumentando as chances de interação inicial.

O capítulo também explorará a preferência visual (2.3), analisando como a atratividade física e as primeiras impressões impactam a decisão dos usuários de prosseguir com uma conexão potencial. A compreensão dos padrões visuais preferidos e sua incorporação em algoritmos é fundamental para aprimorar a experiência dos usuários e aumentar a eficácia dos matches sugeridos.

Por fim, são apresentadas as considerações finais (2.4), resumindo os principais conceitos discutidos e sua relevância para o desenvolvimento de algoritmos de matchmaking mais eficazes e personalizados em aplicativos de namoro. Essa análise teórica visa fornecer uma base sólida para a implementação prática e a avaliação do impacto dessas técnicas nos aplicativos de relacionamento modernos.