\section{Introdução}%
    \subsection{Contexto}
Na era digital, os relacionamentos evoluíram significativamente, com a tecnologia desempenhando um papel central na forma como as pessoas se conectam. Aplicativos de namoro como Tinder, Hinge e Coffee Meets Bagel utilizam IA para analisar grandes volumes de dados comportamentais dos usuários, personalizando as sugestões de parceiros de forma a aumentar as chances de matches bem-sucedidos.

A relevância da IA nesses aplicativos está na sua capacidade de identificar padrões de comportamento e prever compatibilidade entre os usuários. Por exemplo, algoritmos de recomendação, semelhantes aos utilizados por plataformas de streaming como Netflix, são aplicados para oferecer sugestões personalizadas com base no histórico de interações e preferências dos usuários. Essa personalização melhora a experiência do usuário, proporcionando recomendações mais precisas e relevantes \textcolor{blue}{[\cite{Bonilla2023}]}.

Estudos recentes mostram que a IA não apenas sugere parceiros, mas também pode prever a estabilidade e a longevidade dos relacionamentos. Análises de dados comportamentais, como frequência de comunicação e interesses comuns, são fatores que contribuem para a previsão da viabilidade dos relacionamentos. Aplicativos como Blush e Aimm utilizam testes de personalidade e análise de preferências físicas para treinar seus sistemas de IA, prometendo maiores chances de encontrar uma combinação perfeita \textcolor{blue}{[\cite{Finkel2012}]}.

Os benefícios da aplicação de IA em aplicativos de namoro incluem a capacidade de fornecer recomendações mais precisas e personalizadas, aumentando as chances de formar relacionamentos significativos. Além disso, a eficiência das conexões melhora, facilitando encontros mais rápidos e criativos. No entanto, existem desafios importantes, como a proteção da privacidade dos dados dos usuários e a necessidade de evitar vieses nos algoritmos, que podem afetar a equidade nas recomendações \textcolor{blue}{[\cite{Sharabi2022}]}.

Aplicativos como Tinder utilizam IA para analisar interações e preferências dos usuários, ajustando-se dinamicamente ao comportamento do usuário ao longo do tempo. Coffee Meets Bagel foca em conexões através de amigos em comum, utilizando dados de redes sociais para criar um senso de confiança e familiaridade entre os usuários, promovendo interações mais significativas \textcolor{blue}{[\cite{Saban2024}]}.

A aplicação de IA em aplicativos de namoro representa uma área promissora, oferecendo soluções inovadoras para melhorar a eficácia dos relacionamentos virtuais. Contudo, é essencial abordar os desafios éticos e técnicos para garantir que essas tecnologias beneficiem os usuários de maneira justa e segura. A privacidade dos dados, a transparência nos algoritmos e a atualização constante das tecnologias são aspectos cruciais para o sucesso e a aceitação dessas ferramentas.

Ao investigar a aplicação de IA em aplicativos de namoro, este estudo contribui para o entendimento das interações virtuais e oferece insights valiosos para o desenvolvimento de estratégias mais eficazes em plataformas de namoro online. A análise criteriosa de dados e algoritmos atuais visa fornecer uma base sólida para futuras pesquisas e inovações na área. A possibilidade de integrar variáveis psicológicas e sociológicas mais complexas representa uma direção futura promissora para aprimorar ainda mais a precisão e a eficácia dos algoritmos de recomendação.
\subsection{Objetivo}
O presente trabalho de conclusão de curso (TCC) tem como objetivo central explorar e implementar metodologias de inteligência artificial (IA) em um contexto de aplicativo de namoro, com a finalidade de desenvolver um algoritmo de matchmaking mais completo. Este algoritmo não se limitará apenas à recomendação de usuários, mas também à previsão de potenciais usuários para um relacionamento de sucesso e estável. A implementação deste algoritmo será realizada em um contexto real, em um aplicativo de namoro funcional.

\subsection*{Justificativa do Objetivo}

\begin{enumerate}
    \item \textbf{Transformação dos Algoritmos de Matchmaking}\\
    Tradicionalmente, os algoritmos de matchmaking em aplicativos de namoro se baseiam em dados demográficos básicos e preferências superficiais. No entanto, a introdução de IA permite uma análise mais profunda e personalizada das preferências dos usuários. Algoritmos de IA, como o "Most Compatible" do Hinge, utilizam aprendizado de máquina para analisar as preferências e ações dos usuários, sugerindo matches altamente compatíveis. Esta abordagem tem demonstrado aumentar significativamente as chances de matches bem-sucedidos, promovendo conexões mais significativas e duradouras.

    \item \textbf{Previsão de Relacionamentos Estáveis}\\
    A aplicação de IA para prever a estabilidade de relacionamentos vai além da simples correspondência de perfis. Estudos mostram que algoritmos podem ser desenvolvidos para analisar aspectos mais profundos, como comportamentos e interações dos usuários, facilitando previsões sobre a durabilidade e sucesso de um relacionamento. Este avanço é crucial para a evolução dos aplicativos de namoro, proporcionando aos usuários não apenas um parceiro potencial, mas um parceiro com alta probabilidade de um relacionamento duradouro e satisfatório.

    \item \textbf{Implementação em um Contexto Real}\\
    Para validar a eficácia do algoritmo desenvolvido, é essencial sua implementação em um ambiente real. Aplicativos como Tinder e Bumble já utilizam IA para melhorar a experiência do usuário, otimizando fotos de perfil e moderando conteúdo para aumentar a segurança. A implementação em um aplicativo funcional permitirá testar e ajustar o algoritmo em tempo real, garantindo sua eficácia e eficiência no mundo real. Além disso, aplicativos como o Match.com utilizam chatbots com IA para sugerir locais de encontro, mostrando o potencial da IA em oferecer uma experiência de usuário mais completa e personalizada.
\end{enumerate}
\subsection{Organização da Monografia}
\begin{enumerate}
    \item \textbf{Capítulo 2: Padrões Comportamentais}
    \begin{enumerate}
        \item \textbf{Introdução}
        
        Essa seção tem o dever de introduzir os padrões comportamentais mais relevantes para um relacionamento de sucesso. Serão analisados estudos científicos e, além disso, serão investigados padrões comportamentais que são considerados vantajosos e utilizados em aplicativos de relacionamento existentes.

        \item \textbf{Interesses em Comum}
        
        Esta seção irá explorar a influência da personalidade nos relacionamentos. Será discutido como características pessoais, traços psicológicos e interesses em comum podem ser usados para prever a compatibilidade entre usuários. A análise incluirá a revisão de modelos teóricos e estudos empíricos que demonstram a importância da personalidade para a formação de relações duradouras.

        \item \textbf{Conhecidos em Comum (Grafo de Rede Social)}
        
        Nesta parte, a análise se concentrará em como as conexões sociais afetam a probabilidade de sucesso em aplicativos de namoro. Utilizando o conceito de grafos de redes sociais, será discutido o impacto de conhecidos em comum na facilitação de encontros mais confiáveis e com maior potencial de sucesso. Estudos de caso e pesquisas acadêmicas que analisam essa dinâmica serão apresentados.

        \item \textbf{Preferência Visual}
        
        O objetivo desta seção é examinar o papel das preferências visuais no processo de formação de relacionamentos. Será avaliado como os algoritmos podem ser treinados para reconhecer padrões visuais que indicam preferências estéticas e como essas preferências influenciam a atração e a decisão de iniciar um contato.
    \end{enumerate}

    \item \textbf{Capítulo 3: Fundamentação Teórica}
    \begin{enumerate}
        \item \textbf{Convolutional Neural Networks (CNNs)}
        
        Explora o uso de CNNs para extrair características visuais de perfis, melhorando a recomendação de parceiros ao identificar preferências visuais através de imagens.

        \item \textbf{Transformers}
        
        Discute como Transformers utilizam autoatenção para analisar descrições de perfis e mensagens, capturando interesses e compatibilidades em dados sequenciais.

        \item \textbf{Graph Neural Networks (GNNs)}
        
        Aborda como GNNs modelam conexões sociais, propagando informações entre nós para identificar padrões de interação e fortalecer recomendações.

        
    \end{enumerate}
    
    \item \textbf{Capítulo 4: Trabalhos Relacionados}
    \begin{enumerate}
        \item \textbf{Cafe: Predicting Physical Attraction with Deep Learning-Based Systems}
        
        Descreve como o projeto "Cafe" utiliza aprendizado profundo para prever a atratividade física, aplicando CNNs para analisar imagens de perfis em plataformas de namoro.

        \item \textbf{Dataset OkCupid e Instagram Graph API}
        
        Examina como o dataset OkCupid e a API do Instagram Graph podem ser utilizados para aprimorar sistemas de recomendação, focando em agregação de personalidade e social.
        \item \textbf{Graph Neural Networks for Social Recommendation (GraphRecWWW19) and Adaptation}
        
        Analisa o GraphRec, combinando GNNs e atenção para integrar dados de grafos sociais e de itens, com adaptações para incluir traços de personalidade e características visuais.
    \end{enumerate}

    \item \textbf{Capítulo 5: Situação Atual}
    \begin{enumerate}
        \item \textbf{Progresso Atual}
        
        Aqui será apresentado um resumo das atividades realizadas, com ênfase nos resultados obtidos até o momento. Serão destacadas as etapas concluídas e as descobertas mais significativas do projeto.

        \item \textbf{Próximos Passos}
        
        Será identificada e discutida a sequência de etapas futuras necessárias para a conclusão do projeto. Esta seção destacará as prioridades e planos para finalizar o desenvolvimento do algoritmo e sua implementação.

        \item \textbf{Conclusão}
        
        Finalmente, as considerações finais refletirão sobre o trabalho realizado, discutindo as implicações dos resultados para o campo de estudo e propondo direções para pesquisas futuras. A seção também abordará as contribuições do projeto para o avanço dos aplicativos de namoro e suas aplicações práticas.
    \end{enumerate}
\end{enumerate}
