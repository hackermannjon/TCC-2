
\section{Introdução}
Este capítulo tratará dos conceitos teóricos essenciais para a elaboração deste Trabalho de Conclusão de Curso, que busca explorar algoritmos de matchmaking em aplicativos de namoro. Inicialmente, serão discutidos os interesses em comum (2.2), que incluem gostos, valores e atividades compartilhadas entre indivíduos, demonstrando como essas semelhanças influenciam a formação e a durabilidade dos relacionamentos.

Em seguida, o capítulo abordará o conceito de conhecidos em comum (2.3), uma estratégia que explora a rede social dos usuários para identificar conexões mútuas que possam facilitar interações e criar um ambiente de confiança entre os possíveis parceiros. Estudos demonstram que conhecidos em comum podem atuar como uma ponte de confiança, aumentando as chances de interação inicial.

Por fim, capítulo também explorará a preferência visual (2.4), analisando como a atratividade física e as primeiras impressões impactam a decisão dos usuários de prosseguir com uma conexão potencial. A compreensão dos padrões visuais preferidos e sua incorporação em algoritmos é fundamental para aprimorar a experiência dos usuários e aumentar a eficácia dos matches sugeridos.


\section{Importância de Interesses e Hobbies Comuns}
\subsection{Efeito de Similaridade-Atração}
O efeito de similaridade-atração é um fenômeno amplamente estudado na psicologia social que sugere que as pessoas são mais propensas a se sentir atraídas por indivíduos que compartilham características semelhantes, como atitudes, interesses e valores. Essa teoria se fundamenta na ideia de que a similaridade proporciona uma base de entendimento mútuo e validação social, elementos cruciais para o desenvolvimento e manutenção de relações interpessoais bem-sucedidas \textcolor{blue}{[\cite{Berscheid1998}]}.

Os estudos sobre o efeito de similaridade-atração indicam que a semelhança entre indivíduos facilita interações mais agradáveis e reforça a percepção de aceitação e pertencimento. Isso ocorre porque os indivíduos são mais propensos a esperar rejeição por parte de pessoas que são dissimilares, enquanto a interação com semelhantes tende a ser mais confortável e agradável Além disso, a similaridade aumenta a probabilidade de encontros casuais em ambientes sociais comuns, como eventos esportivos ou atividades culturais, onde interesses compartilhados são evidentes \textcolor{blue}{[\cite{Newcomb1961}]}.

Outra explicação para o efeito de similaridade-atração é o reforço social, onde interações com indivíduos semelhantes são mais gratificantes, pois reforçam as próprias crenças e atitudes de cada indivíduo. Esse reforço ocorre tanto em contextos de amizade quanto em relacionamentos românticos, onde a validação mútua fortalece o vínculo \textcolor{blue}{[\cite{Byrne1971}]}. Além disso, pesquisas demonstram que, em plataformas de namoro online, perfis que exibem maior similaridade percebida tendem a receber mais interações positivas.

Por fim, é importante considerar que, embora a similaridade possa inicialmente atrair parceiros, a percepção contínua de semelhança pode ser necessária para manter a atração ao longo do tempo. Discrepâncias percebidas em áreas centrais podem levar a desentendimentos, enquanto a manutenção de um senso de similaridade, mesmo em áreas onde as opiniões podem divergir, ajuda a sustentar relacionamentos saudáveis \textcolor{blue}{[\cite{Montoya2008}]}.

\subsection{Raciocínio Autoessencialista}
O raciocínio autoessencialista é uma abordagem teórica que propõe que a atração por indivíduos semelhantes ocorre devido a uma percepção de essência compartilhada. Esse conceito sugere que as pessoas tendem a categorizar aqueles que compartilham atributos semelhantes como "pessoas como eu", aplicando uma essência comum que facilita a identificação de uma realidade compartilhada \textcolor{blue}{[\cite{Chu2023}]}.

Pesquisas indicam que o raciocínio autoessencialista desempenha um papel fundamental na forma como as pessoas percebem e interpretam a similaridade em relacionamentos interpessoais. Estudos experimentais mostram que indivíduos com crenças autoessencialistas mais fortes experimentam uma maior atração por pessoas semelhantes, pois veem essas semelhanças como indicativas de um entendimento comum do mundo.

A aplicação desse raciocínio em relacionamentos românticos é especialmente relevante, pois ajuda a explicar por que interesses e valores compartilhados podem levar a conexões mais profundas e satisfatórias. Quando as pessoas percebem que compartilham uma essência com seu parceiro, isso reforça a conexão emocional e a compreensão mútua, elementos essenciais para a construção de um relacionamento duradouro.

Além disso, o raciocínio autoessencialista sugere que a similaridade não é apenas uma questão de compartilhamento de características superficiais, mas também de um alinhamento mais profundo nas formas de ver e entender o mundo. Isso pode ser particularmente importante na resolução de conflitos, onde a percepção de uma realidade compartilhada ajuda os parceiros a navegarem por desafios e diferenças de maneira mais harmoniosa.

Essa abordagem também destaca a importância de cultivar uma percepção de similaridade em áreas centrais de um relacionamento, ao mesmo tempo em que reconhece e valoriza as diferenças. Ao aplicar o raciocínio autoessencialista, os aplicativos de namoro podem aprimorar suas recomendações, focando em interesses e valores fundamentais que promovem uma sensação de conexão essencial entre os usuários \textcolor{blue}{[\cite{Montoya2013}]}.

\subsection{Importância do Compartilhamento de Experiências}
A importância do compartilhamento de experiências em relacionamentos interpessoais é uma área de crescente interesse na psicologia social, destacando como essas experiências contribuem para o fortalecimento das conexões emocionais e sociais \textcolor{blue}{[\cite{Berscheid1998}]}.

Estudos demonstram que experiências compartilhadas aumentam a percepção de emoções e sensações comuns, o que por sua vez fortalece a conexão emocional entre as pessoas envolvidas \textcolor{blue}{[\cite{Byrne1971}]}. Isso é evidenciado pela ativação dos circuitos de recompensa no cérebro durante experiências compartilhadas, como assistir a um filme ou participar de atividades conjuntas, indicando que o simples ato de compartilhar pode intensificar a experiência emocional.

Além disso, as experiências compartilhadas fornecem uma base comum que facilita a comunicação e a compreensão mútua. Elas servem como pontos de referência que podem ser revisitados e discutidos, criando uma narrativa compartilhada que fortalece a identidade do relacionamento \textcolor{blue}{[\cite{Montoya2013}]}. O impacto emocional dessas experiências é amplificado quando os indivíduos sentem que suas percepções e reações estão alinhadas com as dos outros, promovendo um senso de pertencimento e validação.

Em um contexto de aplicativos de namoro, incentivar encontros e atividades que promovam o compartilhamento de experiências pode melhorar a eficácia dos algoritmos de matchmaking. Ao conectar usuários com base em interesses e atividades comuns, os aplicativos podem facilitar o desenvolvimento de relações mais profundas e significativas \textcolor{blue}{[\cite{Chu2023}]}.

A promoção de experiências compartilhadas não se limita apenas a eventos agradáveis; enfrentar desafios e superar obstáculos juntos também pode fortalecer o vínculo entre os parceiros. Isso ocorre porque a superação de dificuldades em conjunto reforça a confiança e demonstra a capacidade de colaboração, aspectos essenciais para a resiliência de um relacionamento \textcolor{blue}{[\cite{Montoya2008}]}.


\section{Influência de Conhecidos em Comum}
\subsection{Teoria dos Laços Fracos de Granovetter}
A Teoria dos Laços Fracos, proposta por Mark Granovetter em 1973, é uma das contribuições mais influentes para o estudo das redes sociais. Esta teoria sugere que os laços fracos, que são conexões mais distantes ou menos frequentes, desempenham um papel crucial na difusão de informações e na formação de novas oportunidades dentro de uma rede social. Granovetter argumentou que, enquanto laços fortes (como familiares e amigos próximos) são importantes para o apoio emocional, são os laços fracos que introduzem informações novas e diversas, conectando indivíduos a diferentes redes sociais e possibilitando o acesso a novos grupos sociais.

Em um contexto de relacionamentos, laços fracos, como conhecidos em comum, podem atuar como pontes entre diferentes círculos sociais, aumentando a probabilidade de encontros entre indivíduos que, de outra forma, não teriam se cruzado. A presença de conhecidos em comum pode facilitar a confiança inicial e fornecer um ponto de partida para interações significativas. Estudos empíricos, como os realizados na plataforma LinkedIn, demonstraram que laços fracos são especialmente eficazes para facilitar a mobilidade social e profissional, ilustrando seu potencial para também promover novos relacionamentos pessoais.

Granovetter também destaca que, em ambientes onde a troca de informações é vital, como em economias digitais e setores de alta tecnologia, a importância dos laços fracos é amplificada. Nesse sentido, os aplicativos de namoro podem se beneficiar ao identificar e utilizar laços fracos para recomendar parceiros potenciais, criando oportunidades para conexões que de outra forma não ocorreriam. A teoria dos laços fracos sugere que a introdução de novos contatos através de conhecidos em comum não apenas amplia as redes sociais, mas também enriquece as experiências sociais, oferecendo aos indivíduos uma chance de explorar novas dinâmicas interpessoais \textcolor{blue}{[\cite{Granovetter1973}]}.

\subsection{Teoria do Suporte Social}
A Teoria do Suporte Social é uma estrutura conceitual amplamente estudada que examina como as redes sociais proporcionam apoio emocional, instrumental e informacional aos indivíduos. Essa teoria sugere que a presença de suporte social robusto pode atuar como um amortecedor contra o estresse e promover o bem-estar psicológico e físico. Conhecidos em comum são um componente vital dessas redes de suporte, fornecendo um contexto social que facilita a troca de apoio e validação entre parceiros

Empiricamente, a teoria do suporte social é apoiada por numerosos estudos que demonstram que indivíduos com redes sociais fortes, incluindo conhecidos em comum, relatam maior satisfação e estabilidade em seus relacionamentos. Essas redes oferecem um ambiente de confiança e segurança, essencial para o desenvolvimento de relações íntimas e duradouras. A presença de amigos em comum pode atuar como um mediador em conflitos, oferecendo conselhos imparciais e suporte prático que reforçam os laços entre parceiros.

Além disso, a teoria destaca a importância da reciprocidade nas interações sociais, onde a troca mútua de apoio fortalece as conexões e promove a resiliência do relacionamento. Em um cenário de aplicativos de namoro, a integração de conhecidos em comum pode não apenas aumentar a confiança inicial entre os usuários, mas também criar uma rede de suporte que encoraje interações mais significativas e sustentáveis. O suporte social proporcionado por conhecidos em comum pode, portanto, desempenhar um papel crucial na facilitação e manutenção de relacionamentos estáveis \textcolor{blue}{[\cite{Cohen2004}]}.

\subsection{Teoria do Contágio Social}
A Teoria do Contágio Social explora como comportamentos, emoções e normas são transmitidos através de redes sociais, influenciando os indivíduos de maneira direta e indireta. Conhecidos em comum podem servir como catalisadores para o contágio social, promovendo a adoção de comportamentos positivos e normas sociais que favorecem a formação e o fortalecimento de relacionamentos .

De acordo com a teoria, quando duas pessoas compartilham amigos, há uma maior probabilidade de que atitudes e comportamentos positivos sejam reforçados, criando um ambiente mais propício para o desenvolvimento de laços estáveis. Esse fenômeno pode ser observado em várias áreas, desde a saúde mental até o comportamento organizacional, onde o contágio social ajuda a difundir práticas saudáveis e a promover a coesão social.

Em relacionamentos, o contágio social pode facilitar a convergência de valores e expectativas entre os parceiros, aumentando a compatibilidade e a satisfação. Conhecidos em comum podem atuar como modelos de comportamento, incentivando práticas que fortalecem a relação, como a comunicação aberta e o apoio mútuo. Nos aplicativos de namoro, considerar o contágio social através de conhecidos em comum pode ajudar a identificar matches com maior potencial de sucesso, ao alinhar usuários com normas e comportamentos compatíveis \textcolor{blue}{[\cite{Christakis2007}]}.

\subsection{Aplicativos de Sucesso Utilizando Conhecidos em Comum}
O uso de conhecidos em comum como ferramenta de matchmaking tem se mostrado eficaz em vários aplicativos de namoro de sucesso. Exemplos notáveis incluem Tinder, Hinge e Bumble, cada um integrando essa funcionalidade de maneira a enriquecer a experiência do usuário e promover conexões mais significativas .

Tinder utiliza amigos em comum para aumentar a confiança e facilitar a introdução entre potenciais matches. Através da integração com redes sociais como o Facebook, Tinder oferece aos usuários a capacidade de ver amigos em comum, fornecendo uma base de familiaridade que pode encorajar interações iniciais. Este recurso é reforçado pelo "Tinder Matchmaker", onde amigos podem recomendar perfis uns aos outros, destacando a importância das redes sociais compartilhadas na formação de conexões.

Hinge também adota conhecidos em comum como uma parte central de seu algoritmo de matchmaking. Ao focar em conexões significativas, Hinge permite que os usuários identifiquem amigos compartilhados com possíveis matches, criando uma rede de suporte que pode validar e encorajar novas interações. Este enfoque em profundidade e compatibilidade é parte do que faz de Hinge uma escolha popular para aqueles que procuram relacionamentos sérios e duradouros.

Bumble incorpora conhecidos em comum para promover um ambiente seguro e autêntico. Ao usar conexões sociais compartilhadas para sugerir possíveis matches, Bumble cria uma rede onde a confiança e a familiaridade são fundamentais. Isso é especialmente importante para usuários que buscam interações respeitosas e genuínas, alinhando-se com a missão do aplicativo de empoderar as mulheres e criar conexões mais significativas \textcolor{blue}{[\cite{DatingApps2024}]}.

\section{Preferência Visual}

\subsection{Efeito de Halo}

O efeito de halo é um fenômeno psicológico em que a percepção de uma característica positiva em uma pessoa, como a atração física, influencia a percepção de outras características, como inteligência ou simpatia. No contexto dos aplicativos de namoro, o efeito de halo pode desempenhar um papel crucial na forma como os perfis são avaliados. Pesquisas indicam que perfis com fotos atraentes recebem avaliações mais favoráveis em relação a traços de personalidade, mesmo que essas informações não estejam explicitamente disponíveis \textcolor{blue}{[\cite{jones2021attractiveness}]}.

Estudos mostram que o efeito de halo pode levar a uma sobrevalorização das qualidades percebidas com base na aparência física, facilitando o início de conversas e aumentando a probabilidade de interações subsequentes. Em aplicativos de namoro, isso se traduz na tendência dos usuários de priorizarem perfis visualmente atraentes, acreditando que eles também possuem outras características desejáveis, como confiança e amabilidade.

No entanto, o efeito de halo também apresenta desafios. Enquanto ele pode ajudar a iniciar interações, a manutenção de um relacionamento requer uma percepção mais equilibrada das qualidades reais de uma pessoa. Nos aplicativos de namoro, isso significa que a primeira impressão visual, enquanto poderosa, precisa ser apoiada por uma interação e comunicação genuínas para que um relacionamento significativo e duradouro se desenvolva \textcolor{blue}{[\cite{ramaker2020impact}]}.

\subsection{Hipótese do Matching}

A hipótese do matching sugere que as pessoas tendem a escolher parceiros de nível semelhante de atração física. Esta teoria é amplamente suportada por evidências empíricas, mostrando que casais que se consideram igualmente atraentes têm maior probabilidade de formar relacionamentos duradouros e satisfatórios \textcolor{blue}{[\citemazzella1994effects}]}.

Em aplicativos de namoro, essa hipótese se reflete na forma como os usuários interagem. Estudos demonstram que os indivíduos são mais propensos a iniciar contato com aqueles que percebem estar em um "nível" similar de atratividade, aumentando as chances de reciprocidade e interação positiva. Esta dinâmica é crucial para o sucesso dos aplicativos, onde a atratividade visual muitas vezes serve como o primeiro critério na seleção de parceiros.

A aplicação da hipótese do matching nos algoritmos de matchmaking pode ajudar a criar um ambiente onde os usuários se sintam mais confortáveis e seguros ao explorar potenciais relacionamentos, sabendo que suas preferências visuais estão sendo consideradas. Isso não apenas melhora a satisfação do usuário, mas também contribui para a formação de relacionamentos que têm maior potencial para serem estáveis e gratificantes \textcolor{blue}{[\cite{ramaker2020impact}]}.

\subsection{Teoria da Seleção Sexual}

A teoria da seleção sexual, fundamentada na biologia evolutiva, propõe que a atração física é um indicativo de saúde e fertilidade, características que historicamente aumentaram o sucesso reprodutivo. Essa teoria explica por que a atratividade visual é frequentemente um fator primordial na escolha de parceiros, especialmente em plataformas de namoro, onde a aparência pode ser uma das principais considerações.

Estudos de seleção sexual destacam que características físicas, como simetria facial e proporções corporais, são vistas como atraentes porque sinalizam boa saúde genética e potencial reprodutivo. Nos aplicativos de namoro, isso se traduz em perfis que enfatizam a atratividade física, muitas vezes liderando a escolhas iniciais de interação \textcolor{blue}{[\cite{marcinkowska2014cross}]}.

Embora a seleção sexual priorize a aparência, o desenvolvimento de relacionamentos bem-sucedidos requer que esses fatores visuais sejam apoiados por compatibilidade em outros aspectos, como valores e interesses. Nos aplicativos de namoro, a integração de teorias de seleção sexual em algoritmos pode ajudar a balancear a atração inicial com a compatibilidade a longo prazo, promovendo não apenas encontros baseados na atração física, mas também na qualidade e durabilidade das conexões \textcolor{blue}{[\cite{thornhill2006facial}]}.

\subsection{Aplicativos de Namoro e Preferência Visual}

A maioria dos aplicativos de namoro atuais utiliza preferências visuais como um critério fundamental para filtrar e apresentar usuários de forma personalizada. Este enfoque na aparência reflete a realidade de que a atratividade visual é um fator imprescindível para a implementação de algoritmos de matchmaking. Aplicativos como Tinder, Bumble e Hinge destacam-se ao priorizar fotos de perfil e aparências físicas como elementos-chave nas interações iniciais.

A personalização baseada na atratividade visual não só aumenta o engajamento dos usuários, mas também melhora a experiência geral, pois os usuários se sentem mais satisfeitos com as sugestões que recebem. A preferência visual já é considerada um componente essencial na arquitetura de algoritmos de matchmaking, tornando-se indispensável para a eficácia dos aplicativos de namoro modernos. Ao incorporar a análise visual em seus sistemas, esses aplicativos conseguem criar matches mais atrativos e personalizados, aumentando as chances de sucesso e satisfação dos usuários.