% tex/discussao/dis_05_conclusao_final.tex
\section{Conclusão Final}
\label{sec:discussao_conclusao_final}

\textbf{Síntese da Contribuição:} \\
Este trabalho de conclusão de curso apresentou o desenvolvimento, a implementação e a avaliação do MatchPredict-AI, um motor de recomendação de perfis para plataformas de relacionamento projetado para ser sensível ao contexto do usuário. Através da implementação da arquitetura GraphRec, que opera sobre representações multimodais dos perfis – incluindo features textuais, demográficas, de personalidade, visuais e embeddings sociais derivados via Node2Vec – o estudo demonstrou empiricamente a capacidade do sistema de personalizar suas sugestões com base no histórico de interações do usuário. A metodologia de avaliação, empregando cenários controlados com personas simuladas ("Baseline", "Persona Consistente" e "Persona Inconsistente"), permitiu quantificar e interpretar o impacto do contexto na performance do modelo, validando a hipótese central do projeto.

A principal contribuição deste trabalho reside na aplicação e validação de uma abordagem baseada em Redes Neurais em Grafos para a tarefa de recomendação em um domínio complexo e multimodal. Os resultados não apenas confirmam a eficácia do GraphRec em integrar diversas fontes de informação, mas também destacam a importância fundamental do contexto do usuário para aprimorar a qualidade da personalização. Embora limitações tenham sido identificadas, o estudo oferece uma base sólida e insights valiosos para futuras pesquisas. O MatchPredict-AI representa, portanto, um passo concreto em direção a sistemas de matchmaking mais inteligentes, robustos e adaptativos, com potencial para melhorar significativamente a experiência do usuário em plataformas de relacionamento online.