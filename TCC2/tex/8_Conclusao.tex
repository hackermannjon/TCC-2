\section{Situação Atual}

Até o momento, o projeto tem avançado de forma significativa, com várias etapas concluídas com sucesso. A pesquisa inicial e a revisão bibliográfica foram realizadas, estabelecendo uma base sólida sobre a aplicação de Inteligência Artificial (IA) em aplicativos de namoro. Os conceitos teóricos de Redes Neurais Convolucionais (CNNs), Transformers e Redes Neurais de Grafos (GNNs) foram aprofundados, destacando como essas tecnologias podem melhorar as recomendações de matchmaking.

O algoritmo existente foi analisado e sua arquitetura foi avaliada para identificar áreas que podem ser aprimoradas com a integração de variáveis psicológicas e comportamentais. Foram encontrados conjuntos de dados existentes, como o OkCupid e dados obtidos via API do Instagram Graph, que servirão de referência para o \textit{social aggregation} e \textit{personality aggregation} na implementação do algoritmo adaptado.

\section{Próximos Passos}

O desenvolvimento do projeto seguirá a metodologia do ciclo de vida do desenvolvimento de IA. Este processo será guiado pelos seguintes passos:

\begin{enumerate}
    \item Utilizaremos como referência o dataset do OkCupid para extrair dados relevantes sobre perfis e interações dos usuários, garantindo que a preparação de dados siga as melhores práticas de normalização e limpeza.
    \item A documentação disponível para a API de grafos do Instagram será empregada para integrar dados sociais, permitindo a modelagem de conexões entre usuários e a identificação de influências sociais relevantes para o algoritmo de recomendação.
    \item Aplicaremos como referência de CNN o algoritmo "Cafe" para analisar preferências visuais, utilizando aprendizado por transferência para melhorar a personalização das recomendações com base em atratividade física.
    \item Posteriormente, faremos a adaptação do algoritmo GraphRecWWW19, mantendo a estrutura de \textit{Attention Networks} e a concatenação dos dados normalizados. Esta abordagem integrará múltiplas fontes de informação para aprimorar a precisão e eficácia do sistema de matchmaking.
    \item Finalmente, validaremos e monitoraremos os resultados do algoritmo implementado, utilizando métricas de desempenho para ajustar e otimizar o modelo em tempo real, assegurando melhorias contínuas e satisfação dos usuários.
\end{enumerate}

O monitoramento contínuo será essencial para garantir que o algoritmo funcione conforme esperado em um ambiente real. Será estabelecido um sistema de feedback para coletar dados dos usuários e ajustar o algoritmo em tempo real, assegurando melhorias constantes na experiência do usuário.
\section{Conclusão}

Este trabalho investigou a aplicação de diversos algoritmos de Inteligência Artificial para aprimorar o sistema de matchmaking em aplicativos de namoro, com foco na adaptação do modelo GraphRecWWW19. O estudo explorou como Redes Neurais Convolucionais (CNNs), Transformers e Redes Neurais de Grafos (GNNs) podem ser integrados de maneira sinérgica para melhorar a precisão e eficácia das recomendações, considerando variáveis comportamentais e sociais.

A partir da análise realizada, cada um dos algoritmos mostrou-se essencial para aspectos específicos do sistema de recomendação. As CNNs destacaram-se na análise de preferências visuais, permitindo personalizar recomendações com base em características físicas e atratividade. Os Transformers, por sua vez, foram eficazes na análise das redes de atenção, utilizados no GraphRecWWW19. Finalmente, as GNNs demonstraram grande capacidade de modelar as relações sociais e interações dentro do grafo de usuários, enriquecendo o sistema com insights sobre influências sociais.

A adaptação do GraphRecWWW19, integrando esses três tipos de algoritmos, pode resultar em um modelo de recomendação robusto e abrangente. Essa abordagem buscar aumentar a precisão das recomendações e permitir uma visão mais holística dos usuários, considerando tanto suas preferências individuais quanto o contexto social em que estão inseridos.

Concluímos que a combinação dessas tecnologias representa um uma visão significativa na personalização e efetividade de aplicativos de namoro, promovendo experiências de usuário mais satisfatórias e interações mais significativas.

Além disso, questões éticas e de privacidade dos dados serão desafios importantes a serem abordados. É essencial que o desenvolvimento de tais tecnologias respeite a privacidade dos usuários e assegure que as recomendações sejam justas e imparciais. Com o progresso contínuo e responsável, a aplicação de IA em plataformas de namoro tem o potencial de transformar a forma como as pessoas se conectam e interagem no mundo digital.
